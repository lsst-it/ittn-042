\documentclass[PMO,authoryear,toc]{lsstdoc}
% lsstdoc documentation: https://lsst-texmf.lsst.io/lsstdoc.html
\input{meta}

% Package imports go here.

% Local commands go here.

%If you want glossaries
%\input{aglossary.tex}
%\makeglossaries

\title{IT Priorities Planning}

% Optional subtitle
% \setDocSubtitle{A subtitle}

\author{%
Cristian Silva
}

\setDocRef{ITTN-042}
\setDocUpstreamLocation{\url{https://github.com/lsst-it/ittn-042}}

\date{\vcsDate}

% Optional: name of the document's curator
% \setDocCurator{The Curator of this Document}

\setDocAbstract{%
The following document details the process to assign priorities to IT work
}

% Change history defined here.
% Order: oldest first.
% Fields: VERSION, DATE, DESCRIPTION, OWNER NAME.
% See LPM-51 for version number policy.
\setDocChangeRecord{%
  \addtohist{1}{YYYY-MM-DD}{Unreleased.}{Cristian Silva}
}


\begin{document}

% Create the title page.
\maketitle
% Frequently for a technote we do not want a title page  uncomment this to remove the title page and changelog.
% use \mkshorttitle to remove the extra pages

\section{Introduction} 

The following document outlines the procedure to assign priorities to IT tasks.

\section{Priorities}


The list of current IT priorities can be found in IT-Priorities\footnote{\url{https://confluence.lsstcorp.org/pages/viewpage.action?spaceKey=IT&title=IT+Priorities}}

\subsection{Levels }

Priorities will be categorized into 3 levels.

  - Priority 1 - These tasks must be done, failure of doing this activity will have a big impact on the construction of the observatory, likely blocking other groups. This is the top priority for IT

  - Priority 2 - These tasks should be done. Most of these activities are new features or solutions to problems detected by the IT group, that should be solved during the construction phase. Failure in completing this task will have a small impact on the construction of the observatory, but they won't block the work of other groups.

  - Priority 3 - Features that are nice to have or desirable by the IT group. Failure in completing these activities will not affect the construction of the observatory and won't block other groups. These activities will provide added value to the systems or procedures of the observatory.

To decide between 2 priorities of the same level, a weighting factor will be applied considering if it's a CAP priority, Summit requirement, etc.

\subsection{Epics }

Each priority will have an Epic created in Jira, to follow its progress.

The Epic will be assigned to the IT Manager and the tasks will be assigned to the responsible for the task. The responsible for each priority is listed on the Confluence page

IT work that is not attached to one of the priorities of the group, such as maintenance tasks, user requests, etc. will be added to a maintenance Epic that will be changed every quarter. The current maintenance Epic is listed at the top of the priorities Confluence page.

The schedule of the IT work is listed in the calendar "IT Work"

\subsection{Size }

Priorities will be sized according to the estimated duration of the task.

  - Small: The Epic can be completed in less than a month.

  - Medium: The Epic can be completed between 1 month and 3 months.

  - Big: The Epic can not be completed in less than 3 months.

\subsection{Due Dates }

Prioritization between tasks of the same priority will be given by the due date of it. These dates will be represented as quarters unless there's a specific date to meet.

Tasks with due dates marked as "ASAP", take the top priority of all the tasks, since they represent a massive failure blocking one or many groups. These tasks must

\subsection{Assignment }

Each IT member will have at least 1 Big priority assigned and a maximum of 2 in a calendar year.

\subsection{Documentation }

Priorities levels 1 and 2 must have an ITTN technote associated with it. The technote for priorities level 1 must be created before or during the development of the task. The technote for priorities level 2 can be created after the task is completed but before the end of the calendar year.

Priorities level 3 don't necessarily need to have a technote.


\appendix
% Include all the relevant bib files.
% https://lsst-texmf.lsst.io/lsstdoc.html#bibliographies
\section{References} \label{sec:bib}
\renewcommand{\refname}{} % Suppress default Bibliography section
\bibliography{local,lsst,lsst-dm,refs_ads,refs,books}

% Make sure lsst-texmf/bin/generateAcronyms.py is in your path
\section{Acronyms} \label{sec:acronyms}
\addtocounter{table}{-1}
\begin{longtable}{p{0.145\textwidth}p{0.8\textwidth}}\hline
\textbf{Acronym} & \textbf{Description}  \\\hline

ASAP & As Soon As Possible \\\hline
CAP & Commissioning Activities Planning \\\hline
IT & Information Technology \\\hline
ITTN & Information Technology Technote \\\hline
PMO & Project Management Office \\\hline
\end{longtable}

% If you want glossary uncomment below -- comment out the two lines above
%\printglossaries





\end{document}
